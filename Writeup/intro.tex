\section{Introduction}
\label{sec:intro}

It is estimated that energy wasted in 75,000 US homes in one year is equivalent to the energy loss due to the British Petroleum oil
spill calamity~\cite{XX}. A portion of this energy wastage can be attributed to callous use of appliances---open refrigerator doors,
open microwave doors, televisions and computers switched on when there is no one in the room, and open doors and windows. The other
portion can be attributed to poor wall insulation as well as drafty doors and windows. While poor insulation can be detected using
energy audits, it is more challenging to determine wastage due to careless use of appliances and windows/doors. To determine such
invisible wastages, continuous monitoring using a low-cost non-intrusive sensing system is required.

A plausible solution to the problem is energy metering. Every appliance can be instrumented with an energy meter and analytics on the
energy consumption data can determine appliance (mis-)use. Unfortunately, however, blanketing a home with energy meters can be
prohibitively expensive and cumbersome. Energy disaggregation on data from a single point measurement system is another solution 
whose accuracy is limited if the number of appliances are large and the sampling rate is low~\cite{XX}. To address this problem, in
this paper, we present the design and preliminary evaluation of a low cost thermal imaging system that uses a single low resolution IR
and RGB camera to determine energy consumption hotspots in a room.

Our system, {\IRLeak} comprises two cameras (IR and RGB) mounted on a steerable base. The system scans a room at a very low frequency. It then uses a combination of image stitching and image segmentation to determine clusters of temperature hotspots in the room. The system can overlay panoramic views of the thermal images with the digital images to provide a visualization of the hotspots. It then uses pre-collected temperature profiles of different surfaces (appliances, doors, windows, and walls) to determine which appliance is being used anamolously. The system can also determine the occupancy of the room. {\IRLeak} costs only \$30, and a single camera suffices for a room. Through a preliminary evaluation using an experimental setup in the lab, we show that {\IRLeak} can accurately find open windows and doors, open refrigerator door, open microwave door, and switched on computers and televisions. 

 
 
 %Building structures, openings and leakages have direct effect on the heating and cooling of homes. Often it has been found that the utility bill is exorbitantly high. Maryland has one of the highest bills in USA~\cite{EIA2014}, where the average monthly utility bill is about \$140. Typically utility costs go up in winter and mostly during bouts of bitter cold when more people are likely to stay indoors HVAC usage spikes up. Times like the winter storm which disrupts may result in families to stay indoors for more number of days hence increasing utility consumption. As such proper heating in the house is a requisite and minimizing the heat loss is necessary to reduce utility bills and also for comfort. Structural openings and leakages can result in uneven thermal pockets in a room or house. Slight openings in doors and windows can be major reasons for heat leakages which can be detected using thermal imagery. Apart from that infrared building diagnosis provides detection of - water leaks and their origin - walls, flooring, roof; plumbing issues like blockages; electrical "hot spots" which can cause potential fire hazards; Pest \& rodent nests; Moisture that cannot be physically reached with moisture meters; HVAC problem areas - loose, ill-fitting or disconnected fittings. However diagnosis providers or self-diagnosis using IR cameras needs manual searching of leakages and uses expensive thermal imaging devices.

 %\indent Our system description is somewhat similar to SPOT+~\cite{SPOT,SPOT+} although our intention and the hardware is completely different. The SPOT system consist of a Kinect sensor, a Infrared Sensor and servo motor, along with environment sensors as the main objective of the system is occupancy based comfort management. The system is described as - "\textit{ When a worker enters the work space, the Kinect tracks the worker and sends a skeleton stream to the PC. The PC finds the location of the worker's body center and calculates the rotation angle of the servos. It then communicates with the micro-controller to adjust the angle of the two servos so that the infrared sensor is facing the body center. When the tracked worker is moving, the infrared sensor may not be actually facing towards the worker. Therefore, we introduce a 0.5 second measurement delay into the system. That is, the infrared sensor starts collecting data only when the worker has been standing still for at least 0.5s. The system then estimates the clothing insulation by the clothing surface temperature ..}".

\subsection*{Research Contributions}

The design, implementation, and evaluation of {\IRLeak} presents 
the following research contributions.

\begin{itemize}
	
 \item {\bf A low-cost virtual audit system}. {\IRLeak} is a low cost thermal imaging system that can determine anamolous use of appliances and energy wasted due to open windows and doors. It obviates the need to 
 instrument appliances with individual energy meters. We develop a simple
 image segmentation and processing algorithm that can reliably detect
 temperature hotspots in the room. This provides the ability to collect
 longitudinal data on energy wastage of homes.
 
 \item {\bf Functional prototype and Evaluation}. We have developed a
 fully functional prototype of {\IRLeak} and present preliminary evaluation
 on the accuracy of detecting anomalous appliance use, human occupancy, and detecting open windows. We show, using a laboratory setup,
 that {\IRLeak} is highly accurate inspite of its simplicity.


 %\item We use an IR camera having a coarse resolution of 16 $\times$ 4 pixels. Calibration with the digital camera is a challenge and ground truth evaluation is difficult. From such a coarse resolution thermal image it is difficult to interpret what camera actually captures.

 %\item The system in SPOT+ has a Kinect and IR sensor together. As such the system is quite cumbersome and not suitable for portable deployment. Also the cost goes high up with the Kinect attached.
 \end{itemize}

 %\indent \textbf{Key Contributions:} We believe that our innovations and results provide strong preliminary evidence that such a hybrid model, where low resolution thermal imagery is used to detect the several thermal sources, can prove to be an attractive and practically viable alternative for building leakage and occupancy detection. The key contributions of our work are as follows.

 %\begin{itemize}
 %\item We develop an integrated thermal imaging system which is capable of capturing snippets of thermal images of wall surfaces. The system can rotate automatically in a pre-calibrated angle to capture continuous IR snapshots in small offsets. The system also has a digital camera which captures images of wall surface.

 %\item The second part of the system is the Image Reconstruction Unit. The IR images and the digital image is transferred to the external Image processing unit where the thermal images are stitched to get a panoramic image and then we detect - human body, cold surface and hot surface.
 %\end{itemize}

 %Rest of the paper is organized as follows. In Section 2 we present the overall architecture of the system. In Section 3 the different components of the imaging system and their detailed description.
