Energy wasted in residential and commercial buildings from leaky windows and doors, open refrigerator doors, and unnecessarily switched-on small to medium everyday appliances account for a large fraction of the energy budget of a country. Energy audit systems provide a remedy to this problem where certain types of energy wastage due to leakages and poor insulation can be analyzed and addressed. However, there is a critical need for low cost systems that can {\em continuously} monitor such wastage and provide recommendations to users on how to mitigate such wastage. To this end, in this paper we present a low cost thermal imaging system, {\IRLeak} that can detect open and drafty windows, open refrigerator and microwave doors, and computers left accidentally on, as well as humans in the room. The system comprises a low resolution, 16 $\times$ 4 IR camera and a low cost digital camera mounted on the steerable platform. It automatically scans a room and takes low resolution IR and RGB images at a low frequency. It then uses image stitching and a segmentation algorithm to determine temperature hotspots in a panoramic image. Finally, it uses pre-captured data on the surface temperature of different appliances, walls, and windows to determine anomalous energy consumption. The system obviates the need to equip every appliance with an energy meter to detect energy wastage and costs less than \$30. The preliminary evaluation in a lab setting shows that {\IRLeak} can reliably detect these events leading to significant energy savings.

 %Thermal Imaging helps detect air leakages, uneven heating or cooling pockets, and damped walls in building environments. For detection of such unseen insulation problems, we present a Thermal Imagery system that captures heat signatures of wall surfaces and provides an integrated thermal image of an entire room. In this paper, we build an alternative compared to the expensive commercially available circuit module consisting of an IR camera connected to a micro-controller with I$^2$C communication protocol, by a Raspberry-Pi module, to reduce the cost of existing thermal imaging system by an order of magnitude. The unit is capable of rotating automatically in pre-calibrated angle and capturing surface thermal images of 16 $\times$ 4 resolution which are then run by an Image Stitching technique to produce an integrated thermal image of the wall surfaces. Finally, another camera unit is placed in-situ to capture the digital images of the wall surface to combine with the reconstructed thermal images to get an overall high-resolution thermal layout of the wall surfaces. The proposed system, based on our preliminary studies, is proven to be a cost effective and user friendly system to detect building leakages, damps and irregular heating/ cooling losses.
